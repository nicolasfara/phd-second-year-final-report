% This is samplepaper.tex, a sample chapter demonstrating the
% LLNCS macro package for Springer Computer Science proceedings;
% Version 2.21 of 2022/01/12
%
\documentclass[runningheads]{llncs}
%
\usepackage[T1]{fontenc}
% T1 fonts will be used to generate the final print and online PDFs,
% so please use T1 fonts in your manuscript whenever possible.
% Other font encondings may result in incorrect characters.
%
\usepackage{graphicx}
% Used for displaying a sample figure. If possible, figure files should
% be included in EPS format.
%
% If you use the hyperref package, please uncomment the following two lines
% to display URLs in blue roman font according to Springer's eBook style:
%\usepackage{color}
%\renewcommand\UrlFont{\color{blue}\rmfamily}
%
\begin{document}
%
\title{Contribution Title\thanks{Supported by organization x.}}
%
%\titlerunning{Abbreviated paper title}
% If the paper title is too long for the running head, you can set
% an abbreviated paper title here
%
\author{Nicolas Farabegoli\inst{1}\orcidID{0000-0002-7321-358X}}
%
%\authorrunning{F. Author et al.}
% First names are abbreviated in the running head.
% If there are more than two authors, 'et al.' is used.
%
\institute{
    Alma Mater Studiorum -- Università di Bologna, Cesena FC 47522, Italy
    \email{nicolas.farabegoli@unibo.it}
}
%
\maketitle              % typeset the header of the contribution
%
% \begin{abstract}
% The abstract should briefly summarize the contents of the paper in
% 150--250 words.

% \keywords{First keyword  \and Second keyword \and Another keyword.}
% \end{abstract}
%
%
%
\section{Research Activities and Outcomes}

The second year of my PhD has been characterized by significant advancement in the research on Collective Adaptive Systems (CAS) and their deployment in edge-cloud environments. This year has been particularly productive, marked by a research visit abroad, substantial publications in high-impact venues, and the exploration of novel theoretical foundations that integrate multiple programming paradigms. The research activities have solidified my expertise in the field while opening new research directions through interdisciplinary collaborations.

\subsection{Research Visit at University of St. Gallen}

From March to June 2025, I conducted a research visit at the University of St. Gallen under the supervision of Prof. Guido Salvaneschi. This experience proved to be transformative for my research trajectory, allowing me to delve deep into effect systems and their potential for unifying different programming approaches. During this visit, I studied how effect systems can serve as a theoretical foundation to combine three complementary paradigms: aggregate computing, multitier programming, and choreography.

The collaboration resulted in breakthrough insights on how these seemingly distinct approaches can be unified under a common theoretical framework, enabling more expressive and safer distributed system programming. This work addresses fundamental challenges in developing large-scale distributed systems by providing formal guarantees about system behavior while maintaining the flexibility needed for modern edge-cloud deployments.

\subsection{Scientific Contributions and Publications}

Building upon the foundations established in my first year, the second year has seen significant theoretical and practical advances in my research domain.

\subsubsection{Journal Publications}

The major milestone of this year was the acceptance of our journal paper in ACM Transactions on Internet of Things (ACM TIOT). This publication represents the culmination of the theoretical work developed during my visit to St. Gallen and provides a comprehensive framework for integrating effect systems with distributed computing paradigms. The paper demonstrates how the proposed approach can enhance the safety, expressiveness, and maintainability of collective adaptive systems deployed across edge-cloud infrastructures.

Additionally, two significant journal submissions have been made to Future Generation Computer Systems (FGCS) and IoT Journal, both extending the theoretical and practical contributions of my research in complementary directions.

\subsubsection{Conference Publications}

The year 2025 has been marked by strong conference publications, particularly at COORDINATION 2025, where two papers were accepted:

\begin{enumerate}
\item \textit{"Declarative Deployment Planning for Green Pulverised Collective Computational Systems"} - This work, developed in collaboration with Antonio Brogi, Roberto Casadei, Stefano Forti, and Mirko Viroli, addresses the critical aspect of energy-efficient deployment in distributed systems. The contribution introduces novel declarative approaches for optimizing the environmental impact of collective computational systems while maintaining their functional properties.

\item \textit{"A Demonstrator for Self-organizing Robot Teams"} - This practical contribution, developed with a large collaborative team including Gianluca Aguzzi, Lorenzo Bacchini, and others, demonstrates the real-world applicability of the theoretical frameworks developed in my research through a concrete robotics application.
\end{enumerate}

\subsubsection{Continuation of Previous Work}

Two journal publications from 2024 represent the maturation of work initiated in my first year:

\begin{enumerate}
\item The paper \textit{"Dynamic IoT deployment reconfiguration: A global-level self-organisation approach"} published in Internet of Things demonstrates how global self-organization principles can be applied to IoT deployment scenarios, providing both theoretical foundations and empirical validation.

\item The continuation of collaborative work in multi-drone coordination and federated learning, showing the versatility and broad applicability of the developed approaches.
\end{enumerate}

\section{Academic Activities}

\subsection{Tutoring Activities}

I continued my involvement in tutoring activities for the course PROGRAMMAZIONE AD OGGETTI [cod. 70219] for the Bachelor's Degree in Computer Science and Engineering during the academic year 2024/2025, building upon the experience gained in my first year and further developing my pedagogical skills.

\subsection{International Collaboration and Mobility}

The research visit to the University of St. Gallen represents a significant step in developing international research collaborations. This experience not only contributed to my research outputs but also enhanced my ability to work in international, interdisciplinary teams and exposed me to different research methodologies and perspectives.

\section{Self Evaluation}

The second year of my PhD has exceeded expectations in terms of both research productivity and theoretical advancement. The research visit to St. Gallen proved to be a pivotal experience, leading to novel theoretical insights and opening new research directions. The integration of effect systems with distributed computing paradigms represents a significant conceptual advance that has the potential to influence the field beyond the immediate scope of my thesis.

The successful publication in ACM TIOT and the acceptance of papers at prestigious conferences like COORDINATION demonstrate the maturation of my research and its recognition by the international community. The collaborative nature of much of this work has also strengthened my ability to work effectively in research teams and has established valuable connections for future research endeavors.

From a technical perspective, the theoretical work on effect systems has provided me with deeper insights into programming language theory and type systems, complementing my previous focus on distributed systems and collective intelligence. This broader theoretical foundation positions me well for addressing fundamental challenges in the field.

\section{Next Year Plan}

For the final year of my PhD, I plan to focus on several key objectives:

\begin{enumerate}
\item \textbf{Thesis Completion}: Synthesize the research contributions from the past two years into a coherent doctoral dissertation that demonstrates the theoretical and practical advances achieved in the field of collective adaptive systems.

\item \textbf{Framework Consolidation}: Complete the development of the comprehensive framework that integrates effect systems with aggregate computing, multitier programming, and choreography, providing both theoretical foundations and practical tooling.

\item \textbf{Empirical Validation}: Conduct extensive experimental validation of the proposed approaches across different application domains, with particular emphasis on edge-cloud scenarios and IoT deployments.

\item \textbf{Publication Strategy}: Finalize the submitted journal papers and pursue additional high-impact publications that showcase the mature research contributions.

\item \textbf{Technology Transfer}: Explore opportunities for transitioning research contributions into practical tools and frameworks that can benefit the broader research and industrial communities.

\item \textbf{Career Preparation}: Prepare for post-doctoral career opportunities through continued collaboration, conference presentations, and engagement with the broader research community.
\end{enumerate}

The foundation established in the first two years provides a strong platform for completing a dissertation that makes significant theoretical and practical contributions to the field of distributed systems and collective adaptive systems.


%
% ---- Bibliography ----
%
% BibTeX users should specify bibliography style 'splncs04'.
% References will then be sorted and formatted in the correct style.
%
% \bibliographystyle{splncs04}
% \bibliography{mybibliography}
%
\begin{thebibliography}{8}
\bibitem{ref_article1}
Author, F.: Article title. Journal \textbf{2}(5), 99--110 (2016)

\bibitem{ref_lncs1}
Author, F., Author, S.: Title of a proceedings paper. In: Editor,
F., Editor, S. (eds.) CONFERENCE 2016, LNCS, vol. 9999, pp. 1--13.
Springer, Heidelberg (2016). \doi{10.10007/1234567890}

\bibitem{ref_book1}
Author, F., Author, S., Author, T.: Book title. 2nd edn. Publisher,
Location (1999)

\bibitem{ref_proc1}
Author, A.-B.: Contribution title. In: 9th International Proceedings
on Proceedings, pp. 1--2. Publisher, Location (2010)

\bibitem{ref_url1}
LNCS Homepage, \url{http://www.springer.com/lncs}. Last accessed 4
Oct 2017
\end{thebibliography}
\end{document}
